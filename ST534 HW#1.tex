\documentclass[12pt, letterpaper]{article}
\usepackage[left=2.5cm,right=2.5cm, top=2.5cm, bottom=2.5cm]{geometry}
\usepackage{fancyhdr}
\pagestyle{fancy}
\fancyhead[R]{Flaherty, \thepage}
\renewcommand{\headrulewidth}{2pt}
\setlength{\headheight}{15pt}
\usepackage{lipsum}
\usepackage{amsmath}
\usepackage[makeroom]{cancel}
\usepackage{cancel}
\usepackage{array,polynom}
\newcolumntype{C}{>{{}}c<{{}}} % for '+' and '-' symbols
\newcolumntype{R}{>{\displaystyle}r} % automatic display-style math mode 
\usepackage{xcolor}
\newcommand\Ccancel[2][black]{\renewcommand\CancelColor{\color{#1}}\cancel{#2}}
% Define a custom environment for examples with an indent

\newenvironment{ex}{
	\par\smallskip % Add some vertical space before the example
	\noindent\textit{Example:\hspace{-0.25em}}
	\leftskip=0.5em % Set the left indent to 1em (adjust as needed)
}{
	\par\smallskip % Add some vertical space after the example
	\leftskip=0em % Reset the left indent
}

\newenvironment{nonex}{
	\par\smallskip % Add some vertical space before the example
	\noindent\textit{Non-example:\hspace{-0.25em}}
	\leftskip=0.5em % Set the left indent to 1em (adjust as needed)
}{
	\par\smallskip % Add some vertical space after the example
	\leftskip=0em % Reset the left indent
}
\newcommand{\mymatrix}[1]{
	\renewcommand{\arraystretch}{0.5} % Adjust vertical spacing%
	\setlength\arraycolsep{3pt}       % Adjust horizontal spacing%
	\scalebox{0.90}{                  % Change font size%
		$\begin{bmatrix}
			#1
		\end{bmatrix}$
	}                   
	\renewcommand{\arraystretch}{1.0} % Reset vertical spacing
	\setlength\arraycolsep{6pt}       %Adjust horizontal spacing%
}

\usepackage{amssymb}
\usepackage{bbm}
\usepackage{mathrsfs}
\usepackage[toc]{glossaries}
\usepackage{amsthm}
\usepackage{indentfirst}
\usepackage[utf8]{inputenc}
\usepackage[thinc]{esdiff}
\usepackage{graphicx}
\graphicspath{{./images/}}
\usepackage{subfig}
\usepackage{chngcntr}
\usepackage{placeins}
\usepackage{caption}
\usepackage{float}
\usepackage{comment}
\usepackage{sectsty}
\sectionfont{\fontsize{15}{15}\selectfont}
\usepackage{subcaption}
\setlength\abovedisplayskip{0pt}
\usepackage[hidelinks]{hyperref}
\usepackage[nottoc,numbib]{tocbibind}
\renewcommand{\qedsymbol}{\rule{0.7em}{0.7em}}
\newcommand{\Mod}[1]{\ (\mathrm{mod}\ #1)}
\counterwithin{figure}{section}
\usepackage{centernot}
\usepackage{enumitem}
\theoremstyle{definition}
\newtheorem{exmp}{Example}
\newtheorem{nonexmp}{Non-Example}
\newtheorem{theorem}{Theorem}[section]
\newtheorem{corollary}{Corollary}[theorem]
\newtheorem{definition}{Definition}[section]
\newtheorem{lemma}{Lemma}[theorem]
\numberwithin{equation}{section}
\newcommand{\mydef}[1]{(Definition \ref{#1}, Page \pageref{#1})}
\newcommand{\mytheorem}[1]{(Theorem \ref{#1}, Page \pageref{#1})}
\newcommand{\mylemma}[1]{(Lemma \ref{#1}, Page \pageref{#1})}
\newcommand{\clickableword}[2]{\hyperref[#1]{#2}}

%underscript for operations%
\newcommand{\+}[1]{+_{\scalebox{.375}{#1}}}
\newcommand{\mult}[1]{\cdot_{\scalebox{.375}{#1}}}

%blackboard for letters%
\newcommand{\E}{\mathbb{E}}
\newcommand{\V}{\mathbb{V}}
\newcommand{\R}{\mathbb{R}}
\newcommand{\N}{\mathbb{N}}
\newcommand{\C}{\mathbb{C}}
\newcommand{\Z}{\mathbb{Z}}
\newcommand{\F}{\mathbb{F}}
\newcommand{\K}{\mathbb{K}}
\newcommand{\1}{\mathbbm{1}}

\title{Time Series HW \# 1}
\author{Liam Flaherty}
\date{\parbox{\linewidth}{\centering%
		Professor Martin\endgraf\bigskip
		NCSU: ST546-001\endgraf\bigskip
		August 29, 2024 \endgraf}}

\begin{document}
\maketitle
\thispagestyle{empty}

\newpage\clearpage\noindent


\noindent\textbf{1) \boldmath{ Let $Z_t=2+0.5t+a_t$ where $\left\{a_t\right\}$ is a white noise sequence with mean zero and variance $\sigma_a^2$. Determine the mean and variance functions for the process $\left\{Z_t\right\}$. Is the process stationary?}}
\vspace{\baselineskip}

We can compute the expectation as:
\begin{align*}
	\E(Z_t)&=\E(2+0.5t+a_t)\\
	&=\E(2)+\E(0.5t)+\E(a_t) &&\text{Expectations are linear}\\
	&=2+0.5t &&\text{Only non-constant is $a_t$, and $\E(a_t)=0$}
\end{align*}

We can compute the variance as:
\begin{align*}
	\V(Z_t)&=\V(2+0.5t+a_t)\\
	&=V(a_t) &&\text{Since $\V(c+Y)=\V(Y)$ for constants $c$}\\
	&=\sigma_a^2
\end{align*}

The mean of $Z_t$ changes over time, so by definition, the process can not be stationary.
\vspace{\baselineskip}





\newpage
\noindent\textbf{2) \boldmath{Henceforth let $\tilde{Z_t}=Z_t-\mu$ be a mean zero process with $\left\{a_t\right\}$ a white noise sequence with variance $\sigma_a^2$. For each of the following, determine if the process is stationary and if it is invertible. If it is stationary, compute the ACF and PACF for lags of $k=0,1,2,3,4$.}}
\vspace{\baselineskip}

An AR process $a_t=\pi(B)\widetilde{Z}_t=\sum\limits_{j=0}^{\infty}\pi_j\tilde{Z}_{t-j}$ is invertible if it is absolutely summable, i.e. $\sum\limits_{j=1}^\infty|\pi_j|<\infty$. Since all the below processes have finitely many non-zero coefficients $\pi_i$, all the processes are automatically invertible (a finite sum of finite values is finite).
\vspace{\baselineskip}

The Yule-Walker equations give us an easy way to compute the autocovariances (and thus autocorrelations) of an AR(p) process. We have $\gamma_k=\pi_1\gamma_{k-1}+\cdots+\pi_p\gamma_{k-p}$ for all lags $k\in \N$.
\vspace{\baselineskip}

In general, we can compute the partial autocorrelations as $\phi_{k,k}=\frac{\left|\mymatrix{
		1 & \rho_1 & \cdots & \rho_{k-2} & \rho_1 \\
		\rho_1 & 1 & \cdots & \rho_{k-3} & \rho_2\\
		\vdots & \vdots & \cdots & \ddots & \vdots\\
		\rho_{k-1} & \rho_{k-2} & \cdots & \rho_1 & \rho_k
	}\right|}{\left|\mymatrix{
		1 & \rho_1 & \cdots & \rho_{k-2} & \rho_{k-1} \\
		\rho_1 & 1 & \cdots & \rho_{k-3} & \rho_{k-2}\\
		\vdots & \vdots & \cdots & \ddots & \vdots\\
		\rho_{k-1} & \rho_{k-2} & \cdots & \rho_1 & 1
	}\right|}$. 

The partial autocorrelation at lag 1 is the same as the autocorrelation. For an AR(p) process, the partial autocorrelation of a lag $k>P$ is zero. Of course, if we had a way to write our model as $\tilde{Z}_{t+k}=\phi_{k,1}\tilde{Z}_{t+k-1}+\phi_{k,2}\tilde{Z}_{t+k-2}+\cdots+\phi_{k,k}\tilde{Z}_{t}+a_{t+k}$, then we could simply pick off the autocorrelation as the coefficient to the last term.

\vspace{\baselineskip}
\noindent\textbf{\boldmath{a. $\tilde{Z}_t-0.5\tilde{Z}_{t-1}=a_t$}}
\vspace{\baselineskip}

We can write the process as $(1-0.5B)\tilde{Z}_t=a_t$. Since the root of $(1-0.5B)$ is $2$, which lies outside the complex unit circle, the process is stationary.
\vspace{\baselineskip}

The process is an AR(1) and so the autocorrelation function from the Yule-Walker equation has just one term-- $\rho_k=\pi_1\rho_{k-1}=0.5\rho_{k-1}$. So $\rho_1=0.5, \rho_2=0.25, \rho_3=0.125, \text{ and } \rho_4=0.06125$.
\vspace{\baselineskip}

The partial autocorrelation is 0.5 at a lag of 1 (it is an AR(1) process, so is the same as the autocorrelation), and 0 at every other non-zero $k$.


\vspace{\baselineskip}
\noindent\textbf{\boldmath{b. $\tilde{Z}_t-0.9\tilde{Z}_{t-1}=a_t$}}
\vspace{\baselineskip}

We can write the process as $(1-0.9B)\tilde{Z}_t=a_t$. Since the root of $(1-0.9B)$ is $\frac{10}{9}$, which lies outside the complex unit circle, the process is stationary.
\vspace{\baselineskip}

For the same reasoning as part a, the autocorrelations are $\rho_1=0.9$, $\rho_2=0.81$, $\rho_3=0.729$, and $\rho_4=0.6561$. The partial autocorrelation is 0.9 at a lag of 1, and 0 at other non-zero $k$.

\vspace{\baselineskip}
\noindent\textbf{\boldmath{c. $\tilde{Z}_t-1.3\tilde{Z}_{t-1}+0.4\tilde{Z}_{t-2}=a_t$}}
\vspace{\baselineskip}

We can write the process as $(1-1.3B+0.4B^2)\tilde{Z}_t=a_t$. The roots of $(0.4B^2-1.3B+1)$ are: $\frac{-(-1.3) \pm \sqrt{(-1.3)^2-4(0.4)(1)}}{2(0.4)}=\frac{1.3 \pm \sqrt{1.69-1.6}}{0.8}=\frac{1.3 \pm \sqrt{.09}}{0.8}=\frac{1.3 \pm 0.3}{0.8}=\frac{5}{4}, 2$. These are real roots which both lie outside the complex unit circle, so the process is stationary.
\vspace{\baselineskip}

The autocorrelations can be computed from the Yule-Walker equations with $p=2$. Recalling $\rho_0=1$ and the symmetry of the correlations, we compute:

\vspace{-0.5cm}
\begin{align*}
	\rho_1&=\pi_1\rho_{1-1}+\pi_2\rho_{1-2}=(1.3)(1)+(-0.4)(\rho_1) \implies \rho_1=\frac{1.3}{1+0.4}\approx 0.93\\
	\rho_2&=\pi_1\rho_{2-1}+\pi_2\rho_{2-2}=(1.3)(\rho_1)+(-0.4)(1) \implies \rho_2=\frac{1.3\cdot1.3}{1.4}-0.4\approx 0.81\\
	\rho_3&=\pi_1\rho_{3-1}+\pi_2\rho_{3-2}=(1.3)(\rho_2)+(-0.4)(\rho_1) \implies \rho_3\approx 0.68\\
	\rho_4&=\pi_1\rho_{4-1}+\pi_2\rho_{4-2}=(1.3)(\rho_3)+(-0.4)(\rho_2) \implies \rho_4\approx 0.56
\end{align*} 


\vspace{\baselineskip}

For the partial autocorrelations, we see:

\begin{gather*}
	\phi_{1,1}=\rho_1\approx0.93\\
	\phi_{2,2}=\frac{\left|\mymatrix{1 & \rho_1\\ \rho_1 & \rho_2}\right|}{\left|\mymatrix{1 & \rho_1\\ \rho_1 & 1}\right|}=\frac{\rho_2-\rho_1^2}{1-\rho_1^2}\approx \frac{0.81-0.93^2}{1-0.93}\approx -0.4
\end{gather*} 

Double-checking our work:

\begin{figure}[H]
	\centering
	\includegraphics[width=9cm]{PACFcalc1}
\end{figure}



\newpage
\noindent\textbf{\boldmath{d. $\tilde{Z}_t-1.2\tilde{Z}_{t-1}+0.8\tilde{Z}_{t-2}=a_t$}}
\vspace{\baselineskip}

We can write the process as $(1-1.2B+0.8B^2)\tilde{Z}_t=a_t$. The roots of $(0.8B^2-1.2B+1)$ are: $\frac{-(1.2) \pm \sqrt{(-1.2)^2-4(0.8)(1)}}{2(0.8)}=\frac{-1.2 \pm \sqrt{1.44-3.2}}{1.6}=\frac{-1.2 \pm \sqrt{-1.76}}{1.6}$. We see there are complex roots, and with the help of software see that the modulus of these roots lie outside the complex unit circle, so the process is stationary.

\begin{figure}[H]
	\centering
	\includegraphics[width=6cm]{ST534 Computation}
\end{figure}

The autocorrelations can be computed from the Yule-Walker equations with $p=2$. Recalling $\rho_0=1$ and the symmetry of the correlations, we compute:

\vspace{-0.5cm}
\begin{align*}
	\rho_1&=\pi_1\rho_{1-1}+\pi_2\rho_{1-2}=(1.2)(1)+(-0.8)(\rho_1) \implies \rho_1=\frac{1.2}{1+0.8}=\frac{2}{3}\\
	\rho_2&=\pi_1\rho_{2-1}+\pi_2\rho_{2-2}=(1.2)(\rho_1)+(-0.8)(1) \implies \rho_2=\frac{1.2\cdot2}{3}-0.8=0\\
	\rho_3&=\pi_1\rho_{3-1}+\pi_2\rho_{3-2}=(1.2)(\rho_2)+(-0.8)(\rho_1) \implies \rho_3=-0.8\frac{2}{3}\approx -0.53\\
	\rho_4&=\pi_1\rho_{4-1}+\pi_2\rho_{4-2}=(1.2)(\rho_3)+(-0.8)(\rho_2) \implies \rho_4\approx -0.64
\end{align*} 

For the partial autocorrelations, we see:

\begin{gather*}
	\phi_{1,1}=\rho_1=\frac{2}{3}\\
	\phi_{2,2}=\frac{\left|\mymatrix{1 & \rho_1\\ \rho_1 & \rho_2}\right|}{\left|\mymatrix{1 & \rho_1\\ \rho_1 & 1}\right|}=\frac{0-\frac{2}{3}^2}{1-\frac{2}{3}^2}= \frac{-\frac{4}{9}}{\frac{5}{9}}=\frac{-4}{5}
\end{gather*} 

Double-checking our work:

\begin{figure}[H]
	\centering
	\includegraphics[width=9cm]{PACFcalc2}
\end{figure}









\newpage
\noindent\textbf{\boldmath{3) Compute the AR and MA processes for the below representations.}}

\vspace{\baselineskip}
\noindent\textbf{\boldmath{a. Find the AR representation of the MA(1) process $\tilde{Z}_t=a_t-0.4a_{t-1}$.}}
\vspace{\baselineskip}

In general, the MA representation is $\tilde{Z}_t=\psi(B)a_t$ and AR representation is $\pi(B)\tilde{Z}_t=a_t$ for some functions $\psi$ and $\pi$ of $B$. Left multiplying the AR representation by $\psi(B)$, we see that $\psi(B)\pi(B)\tilde{Z}_t=\psi(B)a_t=\tilde{Z}_t$, or that $\psi(B)=\pi^{-1}(B)$.
\vspace{\baselineskip}

Applying this to the problem at hand, $\psi(B)=(1-0.4B)$ and so, by the formula for a geometric series (since $0.4<1$), $\pi(B)=(1-0.4B)^{-1}=\sum\limits_{n=0}^{\infty}0.4B^n$. Our AR representation is then $\tilde{Z}_t+0.4\tilde{Z}_{t-1}+0.16\tilde{Z}_{t-2}+\cdots=a_t$, or more compactly $a_t=\sum\limits_{n=0}^{\infty}0.4^n\tilde{Z}_{t-n}$.

\vspace{\baselineskip}
\noindent\textbf{\boldmath{b. Find the MA representation of the AR(2) process $\tilde{Z}_t=0.2\tilde{Z}_{t-1}+0.4\tilde{Z}_{t-2}+a_t$.}}
\vspace{0.05cm}

For notational ease we can rewrite the above as $\tilde{Z}_t-0.2\tilde{Z}_{t-1}-0.4\tilde{Z}_{t-2}=a_t$ and then read off $\pi(B)$ as $(1-0.2B-0.4B^2)$. Here we hit a dead-end since we can't factor the polynomial into the form $(1-xB)(1-yB)$ and use the same geometric series trick in part a. Instead, we try substitution. 
\vspace{\baselineskip}

First, we substitute for $\tilde{Z}_{t-1}$ to eliminate the $\tilde{Z}_{t-1}$ term and add a $a_{t-1}$ term in our representation:

\vspace{-0.5cm}
\begin{align*}
	\tilde{Z}_t&=0.2[0.2\tilde{Z}_{t-2}+0.4\tilde{Z}_{t-3}+a_{t-1}]+0.4\tilde{Z}_{t-2}+a_t\\
	&=0.04\tilde{Z}_{t-2}+0.08\tilde{Z}_{t-3}+0.2a_{t-1}+0.4\tilde{Z}_{t-2}+a_t\\
	&=0.44\tilde{Z}_{t-2}+0.08\tilde{Z}_{t-3}+a_t+0.2a_{t-1}
\end{align*}

Next, we substitute for $\tilde{Z}_{t-2}$ for the same reason:


\vspace{-0.5cm}
\begin{align*}
	\tilde{Z}_t&=0.44[0.2\tilde{Z}_{t-3}+0.4\tilde{Z}_{t-4}+a_{t-2}]+0.08\tilde{Z}_{t-3}+a_t+0.2a_{t-1}\\
	&=0.088\tilde{Z}_{t-3}+0.176\tilde{Z}_{t-4}+0.44a_{t-2}+0.08\tilde{Z}_{t-3}+a_t+0.2a_{t-1}\\
	&=0.168\tilde{Z}_{t-3}+0.176\tilde{Z}_{t-4}+a_t+0.2a_{t-1}+0.44a_{t-2}
\end{align*}

To see where we are going, we can substitute one more time:

\vspace{-0.5cm}
\begin{align*}
	\tilde{Z}_t&=0.168[0.2\tilde{Z}_{t-4}+0.4\tilde{Z}_{t-5}+a_{t-3}]+0.176\tilde{Z}_{t-4}+a_t+0.2a_{t-1}+0.44a_{t-2}\\
	&=0.0336\tilde{Z}_{t-4}+0.0672\tilde{Z}_{t-5}+0.168a_{t-3}+0.176\tilde{Z}_{t-4}+a_t+0.2a_{t-1}+0.44a_{t-2}\\
	&=0.2096\tilde{Z}_{t-4}+0.0672\tilde{Z}_{t-5}+a_t+0.2a_{t-1}+0.44a_{t-2}+0.168a_{t-3}
\end{align*}

Continuing in this fashion, we can continually add an $a_{t-n}$ term. Carefully backtracking the calculation of the $\alpha_{t-3}$ term, we see it is the coefficient of the $\tilde{Z}_{t-3}$ term, which itself  is the sum of the product of 0.44 (the $a_{t-2}$ term) and 0.2 with 0.08 (the product of 0.2-- the $a_{t-1}$ term-- and 0.4). So generically, for $n\geq 3$, we have $\psi_{t-n}=0.2\psi_{t-n+1}+0.4\psi_{t-n+2}$. Compactly, we have $\tilde{Z}_{t}=a_t+0.2a_{t-1}+0.44a_{t-2}+\sum\limits_{j=3}^{\infty}(0.2\psi_{t-j+1}+0.4\psi_{t-j+2})$. 











\newpage
\noindent\textbf{\boldmath{4) Consider the AR(3) process $(1-0.4B)(1-0.2B+0.6B^2)\tilde{Z}_t=a_t$. Let $\sigma_a^2=1$. Determine the roots of $\phi(B)=0$ and then answer the following: Is the process $\tilde{Z}_t$ stationary? Invertible? Why or why not? If the process is stationary, determine its autocorrelation function for integer values of $k$ (you may give recursive equations for $\rho_k$ for $k>3$).}}
\vspace{\baselineskip}

The roots of $\phi(B)$ are $\frac{5}{2}$ (read off from the first factor) and about $0.16\pm 1.28i$ from the quadratic factor (as calculated by software, see below). Since all these roots have a modulus outside the complex unit circle, the process is stationary. Any AR(P) process  with finite P is invertible as explained in question 2.

\begin{figure}[H]
	\centering
	\includegraphics[width=8cm]{ST534 Computation1}
\end{figure}

To compute the autocorrelations, we can expand out the product, write our process in full, and then use the Yule-Walker equations to compute the correlations. We have $\pi(B)=1-0.2B+0.6B^2-0.4B+0.08B^2-0.24B^3$ or $1-0.6B+0.68B^2-0.24B^3$. In full, our process is $\tilde{Z}_t=0.6\tilde{Z}_{t-1}-0.68\tilde{Z}_{t-2}+0.24\tilde{Z}_{t-3}+a_t$. We calculate as follows:

\vspace{-0.5cm}
\begin{align*}
	\rho_1&=\pi_1\rho_{1-1}+\pi_2\rho_{1-2}+\pi_{3}\rho_{1-3}=(0.6)\rho_0+(-0.68)\rho_1+(0.24)\rho_2=(0.6)+(-0.68)\rho_1+(0.24)\rho_2\\
	\rho_2&=\pi_1\rho_{2-1}+\pi_2\rho_{2-2}+\pi_{3}\rho_{2-3}=(0.6)\rho_1+(-0.68)\rho_0+(0.24)\rho_1=(0.84)\rho_1+(-0.68)\\
	\rho_3&=\pi_1\rho_{3-1}+\pi_2\rho_{3-2}+\pi_{3}\rho_{3-3}=(0.6)\rho_2+(-0.68)\rho_1+(0.24)\rho_0=(0.6)\rho_2+(-0.68)\rho_1+(0.24)
\end{align*}

Substituting the second equation into the first, we see:

\vspace{-0.5cm}
\begin{align*}
	\rho_1&=(0.6)+(-0.68)\rho_1+(0.24)\big((0.84)\rho_1+(-0.68)\big)=0.4368-0.4784\rho_1
\end{align*}

So $\rho_1=\frac{0.4368}{1+0.4784} \approx 0.295$, and then $\rho_2=0.84(\frac{0.4368}{1.4784})-0.68\approx -0.432$, and finally $\rho_3=0.6(0.84(\frac{0.4368}{1.4784})-0.68)+(-0.68)(\frac{0.4368}{1.4784})+0.24 \approx -0.22$. For $k>3$, we get the recursion $\rho_k=0.6\rho_{k-1}-0.68\rho_{k-2}+0.24\rho_{k-3}$. As always, we get the negative ``lags" by appealing to symmetry; $\rho_k=\rho_{-k}$.
\vspace{\baselineskip}

Double checking our work:

\begin{figure}[H]
	\centering
	\includegraphics[width=6cm]{AR Calc}
\end{figure} 


\end{document}